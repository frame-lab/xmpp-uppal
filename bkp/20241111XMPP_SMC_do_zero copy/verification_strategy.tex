\subsection{Verification Strategy}
This section outlines our systematic approach to verifying real-time systems using UPPAAL. We present a comprehensive verification strategy that addresses the challenges of ensuring correctness, particularly with respect to timing properties, for complex real-time systems.

\subsubsection{Property Identification}
The first step in our verification strategy is to identify the properties that the system must satisfy. We categorize properties into several types: \textbf{Safety Properties}, which are assertions that "bad things never happen," including examples like mutual exclusion, bounds on resource usage, and absence of deadlock; \textbf{Liveness Properties}, assertions that "good things eventually happen," such as guaranteed service, progress, and fairness; \textbf{Time-Bounded Properties}, assertions about timing constraints like response time bounds, execution time limits, and periodicity requirements; and \textbf{Reachability Properties}, assertions about the existence of paths to specific system states, useful for verifying that certain operations or configurations are possible. For each property, we provide a formal specification in UPPAAL's query language, a natural language description, the rationale for its importance, and the system requirements it addresses.

\subsubsection{Property Formalization}
We formalize properties using UPPAAL's query language, which is based on a subset of Computation Tree Logic (CTL). The main types of properties we consider include: \textbf{Reachability} (\texttt{E<> $\varphi$}), meaning there exists a path where $\varphi$ eventually holds (e.g., \texttt{E<> (Process.Done)} - "It is possible for the process to complete."); \textbf{Safety} (\texttt{A[] $\varphi$}), meaning for all paths, $\varphi$ always holds (e.g., \texttt{A[] (x <= 10)} - "The value of x is always at most 10."); \textbf{Liveness} (\texttt{A<> $\varphi$}), meaning for all paths, $\varphi$ eventually holds (e.g., \texttt{A<> (Request -> Response)} - "Every request is eventually responded to."); and \textbf{Leads-to} (\texttt{$\varphi$ --> $\psi$}), meaning whenever $\varphi$ holds, $\psi$ will eventually hold (e.g., \texttt{Process.Ready --> Process.Processing} - "Whenever the process is ready, it will eventually start processing."). For time-bounded properties, we use clock variables and constraints, such as \texttt{A[] (Process.Start imply Process.clock <= 5)} - "The process always starts within 5 time units," and \texttt{Process.Request --> (Process.Response and Process.responseTime <= 10)} - "Every request is responded to within 10 time units."

\subsubsection{Verification Plan}
We develop a structured verification plan that outlines the order and strategy for property verification, adhering to several principles: \textbf{Progressive Verification}, starting with basic properties before moving to more complex ones; \textbf{Incremental Model Building}, beginning with simplified models and progressively adding complexity, verifying properties at each step; \textbf{Property Dependencies}, identifying dependencies between properties and verifying them in a logical order; and \textbf{Critical Path Analysis}, prioritizing properties that are on the critical path for system correctness. The verification plan includes the sequence of properties to verify, the model configurations for each verification, expected outcomes and success criteria, and strategies for addressing potential failures.

\subsubsection{Verification Techniques}
We employ various verification techniques to handle the complexity of real-time system verification: \textbf{Compositional Verification}, verifying properties of individual components before the complete system to reduce state space; \textbf{Assume-Guarantee Reasoning}, verifying a component assuming certain environmental behaviors and then verifying those assumptions; \textbf{Abstraction and Refinement}, using abstract models for initial verification and refining them as needed; \textbf{Symmetry Exploitation}, identifying system symmetries to reduce verification effort; and \textbf{Parameter Sweeping}, verifying properties across a range of parameter values to identify boundaries.

\subsubsection{Handling State Space Explosion}
State space explosion is a major challenge in model checking, and we mitigate it through several strategies: \textbf{Variable Scoping}, minimizing variable scope and lifetime; \textbf{Committed Locations}, using them to model atomic action sequences and reduce interleaving; \textbf{State Space Reduction Options}, utilizing UPPAAL's built-in techniques like symmetry reduction; \textbf{Abstraction Techniques}, applying domain-specific abstractions to reduce model complexity while preserving properties; and \textbf{Bounded Verification}, checking challenging properties up to a certain depth or time limit.

\subsubsection{Counterexample Analysis}
When verification fails, UPPAAL provides counterexamples that demonstrate property violations. Our approach to analyzing these includes: \textbf{Trace Simulation}, replaying the counterexample trace in UPPAAL's simulator; \textbf{Root Cause Analysis}, identifying the fundamental cause of the violation; \textbf{Classification}, categorizing counterexamples by issue type; and \textbf{Generalization}, determining if the counterexample is an isolated case or a general pattern. Based on this analysis, we either refine the model to fix modeling errors, modify the system design, adjust property specifications, or add auxiliary variables for more insight.

\subsubsection{Performance Optimization}
Verification performance is critical, and we optimize it through: \textbf{Model Simplification}, simplifying irrelevant model aspects; \textbf{Memory Management}, configuring UPPAAL's memory settings based on hardware; \textbf{Verification Order}, verifying properties to allow early issue detection and model refinement; \textbf{Distributed Verification}, using UPPAAL's capabilities across multiple nodes for complex models; and \textbf{Incremental Verification}, focusing verification efforts on affected parts when modifying the model.

\subsubsection{Documentation and Reporting}
Throughout the verification process, we maintain comprehensive documentation for traceability and reproducibility. This includes: \textbf{Property Verification Results}, detailing whether each property was satisfied or violated; \textbf{Model Versions}, tracking changes to the model with rationales; \textbf{Verification Performance}, recording metrics like time and memory usage; and \textbf{Unresolved Issues}, documenting any properties that could not be verified due to limitations. This documentation forms the basis for a verification report summarizing the scope, successfully verified properties, identified issues and their resolutions, limitations, and recommendations for system improvements or further verification efforts.

\subsubsection{Strategy Summary}
Our verification strategy provides a systematic approach to ensuring the correctness of real-time systems with respect to critical properties, particularly timing constraints. By combining formal verification techniques with practical strategies for managing complexity, we address the inherent challenges of real-time system verification. The strategy is adaptable to different types of systems and can be tailored to focus on specific concerns or requirements.