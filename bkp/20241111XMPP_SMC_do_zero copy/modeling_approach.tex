
\subsection{Modeling Approach}
This section details our systematic approach to modeling real-time systems using timed automata within the UPPAAL framework. Our methodology carefully balances model fidelity with the feasibility of analysis, aiming to effectively address the inherent challenges posed by formal verification for complex real-time systems \cite{clarke2018handbook,baier2008principles}.

\subsubsection{System Abstraction}
The initial step in our approach involves developing appropriate  abstractions  of the system under verification. Effective abstraction is crucial for managing the state-space explosion problem while preserving the properties of interest \cite{clarke2007abstraction,bryant1986graph}. Our abstraction process adheres to several key principles. We first identify and model only the  Relevant Timing Aspects  that are pertinent to the properties being verified \cite{alur1994theory,kopetz2022real}, including critical deadlines, response time requirements, and timing dependencies between components. Next, we apply  Component Decomposition , breaking down the system into individual components that can be modeled as distinct timed automata, which facilitates a modular approach to system modeling \cite{larsen1997uppaal,bengtsson1996uppaal}. For  Data Abstraction , we simplify data operations that do not impact timing behavior, replacing complex data structures and algorithms with more streamlined representations when suitable \cite{clarke2007abstraction,beyer2022software}. Finally, we engage in  Environment Modeling , creating explicit models of the environment with which the system interacts, specifically focusing on the timing characteristics of environmental inputs and outputs \cite{alur2000timed,bengtsson2003timed}. This abstraction process necessitates both domain expertise and a thorough understanding of the system's requirements. The ultimate goal is to produce a model that accurately reflects the system's essential timing behavior while remaining amenable to automated verification \cite{woodcock2009formal,baier2008principles}.

\subsubsection{Component Identification and Interfaces}
We identify system components based on their functional boundaries and established interaction patterns \cite{kopetz2022real,hoare1978communicating}. Each component is meticulously characterized by its  Internal State , which includes the component's local variables and clocks that track its internal state and timing \cite{alur1994theory}. Its  External Interface  defines the channels and shared variables through which the component interacts with other components \cite{hoare1978communicating,roscoe1998theory}.  Timing Constraints  specify limitations on the component's behavior, such as minimum and maximum execution times, deadlines, and periodicity \cite{kopetz2022real,buttazzo1997hard}. Furthermore, we define  Error Handling  mechanisms, outlining the component's response to exceptional conditions like timeouts and communication failures \cite{scherer2006synchronization,reghenzani2023software}. Component interfaces are carefully designed to minimize coupling while ensuring that all necessary interactions are fully captured \cite{broy2010seamless}, which promotes modularity and simplifies the refinement of individual components without impacting the integrity of the overall model.

\subsubsection{Template Design in UPPAAL}
For each identified component, we develop a  UPPAAL template  that models its behavior as a timed automaton \cite{larsen1997uppaal,bengtsson1996uppaal}. Our template design adheres to specific guidelines.  State Representation  involves using locations within the template to represent distinct operational states of the component, with invariants expressing crucial timing constraints \cite{alur1994theory,bengtsson2003timed}.  Transitions  are depicted as edges between locations, representing events or actions, with guards specifying when these transitions can occur and updates defining their effects \cite{larsen1997uppaal,david2015uppaal}. We utilize  Clock Variables  to accurately model timing aspects, such as execution durations, waiting times, and timeouts \cite{alur1994theory,alur1999timed}.  Data Variables , including integer and boolean types, model the component's data state, with updates on transitions representing operations on this state \cite{larsen1997uppaal,david2015uppaal}. Finally,  Parameterization  of templates allows for the creation of multiple instances or facilitates the exploration of different configuration options \cite{bengtsson1996uppaal,david2015uppaal}. Special UPPAAL features, such as committed and urgent locations, are judiciously employed to model atomic sequences of actions and to prevent unnecessary time delays \cite{larsen1997uppaal,david2015uppaal}.

\subsubsection{Synchronization and Communication}
Components within a real-time system frequently need to communicate and synchronize their actions \cite{hoare1978communicating,roscoe1998theory}. We model these interactions using several mechanisms.  Binary Synchronization  is achieved through channels with matching send (!) and receive (?) actions, which model one-to-one communication between components \cite{larsen1997uppaal,hoare1978communicating}.  Broadcast Synchronization  utilizes broadcast channels, allowing one component to trigger actions in multiple receivers simultaneously \cite{bengtsson1996uppaal,david2015uppaal}.  Shared Variables , specifically global variables accessible by multiple automata, enable data sharing between components \cite{roscoe1998theory,larsen1997uppaal}. Additionally,  Urgent Channels  are employed for synchronizations that must occur without any time delay, ensuring that the system does not idle when communication is possible \cite{larsen1997uppaal,bengtsson2003timed}. The specific choice of synchronization mechanism is determined by the nature of the interaction and the system's requirements. We generally prefer binary synchronization for explicit handshaking protocols and broadcast synchronization for events that affect multiple components concurrently \cite{hoare1978communicating,david2015uppaal}.

\subsubsection{Modeling Time-Critical Behaviors}
Particular attention is devoted to modeling time-critical behaviors that are essential for the system's correct operation \cite{kopetz2022real,buttazzo1997hard}. We model  Periodic Tasks  using a combination of invariants and guards that enforce regular execution patterns \cite{sha2004real,liu1973scheduling}. For instance, a task with period P can be modeled with a clock that resets after each execution and a guard that enables the next execution when the clock reaches value P.  Deadline Enforcement  is achieved by modeling deadlines using invariants on locations, which compel the system to exit a state before a deadline is violated \cite{alur1994theory,buttazzo1997hard}. For example, a task with deadline D can be modeled with an invariant $x \leq D$ on its execution location, where $x$ is a clock reset at the task's initiation.  Resource Contention  is modeled using additional state variables and mutual exclusion patterns \cite{scherer2006synchronization,sha2004real}. A semaphore, for example, can be represented as an integer variable with acquire and release operations that decrement and increment its value. Finally,  Error Recovery  mechanisms, such as timeouts and retransmissions, are modeled with additional transitions that are enabled when normal operation fails or times out \cite{kopetz2022real,safari2022survey}. These patterns provide a robust foundation for modeling common time-critical behaviors in a manner conducive to formal verification \cite{clarke2018handbook,sha2004real}.

\subsubsection{Model Validation}
Before proceeding to formal verification, we rigorously validate the model to ensure its accurate representation of the system under study \cite{baier2008principles,binder2000testing}. This involves  Simulation , where we utilize UPPAAL's simulator to explore different execution scenarios and compare the model's behavior against the expected system behavior \cite{larsen1997uppaal,david2015uppaal}. We also perform  Sanity Checking  by verifying basic properties that the model should inherently satisfy, such as the absence of deadlocks and the reachability of key states \cite{clarke2018handbook,david2015uppaal}. The model undergoes  Review with Domain Experts  to ensure it accurately captures the essential aspects of the system \cite{woodcock2009formal,broy2010seamless}. Finally, based on validation results, the model undergoes  Incremental Refinement  iteratively to address any discrepancies or missing features \cite{back1990refinement,broy2010seamless}. This comprehensive validation process aids in identifying modeling errors early, thereby significantly increasing confidence in the subsequent verification results \cite{binder2000testing,woodcock2009formal}.

\subsubsection{Scalability Considerations}
As models increase in complexity, managing the state space presents an ever-growing challenge \cite{bryant1986graph,clarke2018handbook}. We employ several techniques to enhance scalability.  Compositional Modeling  involves designing components with well-defined interfaces that can be verified individually before being integrated into a complete system \cite{baier2008principles,hoare1978communicating}. We also use  Abstraction Refinement , beginning with highly abstract models and gradually refining them, focusing on areas where properties fail or more detail is required \cite{clarke2007abstraction,back1990refinement}.  Symmetry Reduction  identifies and exploits symmetries within the system to reduce the effective state space \cite{clarke2018handbook,baier2008principles}; for example, if multiple identical components exist, we might verify properties for a representative subset.  Variable Scoping  minimizes the use of global variables and carefully manages variable lifetimes to reduce the state space dimension \cite{bryant1986graph,mcmillan1993model}. Lastly,  Committed Locations  are utilized to model atomic sequences of actions, which reduces the interleaving of transitions and consequently the state space \cite{larsen1997uppaal,bengtsson2003timed}. These techniques are applied selectively, based on the specific characteristics of the system and the properties undergoing verification \cite{clarke2018handbook,baier2008principles}.

\subsubsection{Experimental Design}
To systematically explore the behavior of the modeled system, we meticulously design experiments that cover various operational scenarios \cite{baier2008principles,binder2000testing}. We incorporate  Parameter Variation  by altering key parameters, such as timing constants, resource capacities, and component configurations, to investigate the system's behavior under diverse conditions \cite{david2015uppaal,bulychev2012uppaal}.  Fault Injection  involves deliberately introducing faults, such as message losses, delays, and component failures, to rigorously test the system's robustness and error recovery mechanisms \cite{reghenzani2023software,safari2022survey}. We also perform  Load Variation , altering the workload on the system by changing the frequency of inputs, the number of clients, or the complexity of operations \cite{bril2008execution,sha2004real}. Finally,  Property Coverage  ensures that experiments are specifically designed to test each property of interest, with a focus on boundary conditions and edge cases \cite{baier2008principles,clarke2018handbook}. Each experiment is accompanied by a defined set of formal properties that capture the expected system behavior under the specified experimental conditions \cite{clarke2018handbook,bengtsson2003timed}.

\subsubsection{Documentation and Traceability}
Throughout the entire modeling process, we maintain comprehensive documentation to ensure clear traceability between the model and the system requirements \cite{woodcock2009formal,broy2010seamless}.  Component Documentation  provides detailed information for each template, including its purpose, assumptions, parameters, and behavioral guarantees \cite{larsen1997uppaal,broy2010seamless}.  Property Documentation  links each property to the specific system requirements it addresses, alongside a clear explanation of its significance \cite{clarke2018handbook,woodcock2009formal}. A  Model Evolution History  tracks all changes made to the model, complete with justifications for design decisions and abstractions \cite{back1990refinement,woodcock2009formal}. Lastly,  Verification Results  are thoroughly documented, including successful verifications, generated counterexamples, and performance metrics \cite{clarke2018handbook,baier2008principles}. This meticulous documentation facilitates model maintenance, enables knowledge transfer, and provides a solid basis for justifying the verification results \cite{woodcock2009formal,broy2010seamless}.

\subsubsection{Methodology Summary}
Our modeling approach provides a systematic framework for creating timed automata models of real-time systems that are both faithful to the system's essential behavior and amenable to verification \cite{alur1994theory,bengtsson2003timed}. By emphasizing abstraction, modularity, and scalability, we effectively address the inherent challenges of formal verification for complex systems \cite{clarke2018handbook,woodcock2009formal}. This adaptable approach can be tailored to different types of real-time systems and can be focused on specific properties of interest \cite{kopetz2022real,baier2008principles}.
